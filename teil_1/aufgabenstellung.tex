\chapter{Aufgabenstellung}\label{ch:aufgabenstellung}
...


\section{Ausgangslage}\label{sec:ausgangslage}
Der \textit{Getting Started Wizard} ist ein stark vereinfachter Config Editor in Airlock SaaS, mit welchem unsere Kunden möglichst schnell und einfach eine funktionsfähige Konfiguration für ihr Tenant IAM erstellen können.
Eine solche Konfiguration beinhaltet sensible Daten, sogenannte \textit{Secrets}.
Ein Secret bezeichnet vertrauliche Informationen wie Schlüssel, Passwörter, Tokens, oder Zertifikate.
Der Schutz dieser Daten vor unbefugtem Zugriff ist entscheidend.
Eine besondere Sicherheitsmassnahme in Airlock SaaS ist daher, dass Secrets separat in einem Vault, einer speziell für Secrets vorgesehenen Datenbank, und nicht in der Applikationsdatenbank abgespeichert werden.

Für den generellen Use Case von Airlock SaaS sind diese Secrets sowie die Integration mit einem Vault vollständig implementiert.
Im Getting Started Wizard hingegen fehlt diese Integration und Secrets werden nicht speziell berücksichtigt.

Die Herausforderung innerhalb dieses Projekts besteht darin, den sicheren Umgang mit Secrets nahtlos in den Getting Started Wizard zu integrieren.

\section{Detaillierte Aufgabenstellung}\label{sec:detaillierte-aufgabenstellung}
\subsection{Analyse}\label{subsec:detailierte-aufgabenstellung-anayse}
Der Kandidat analysiert die existierende Standardkonfiguration sowie die bestehenden Konfigurationsmöglichkeiten auf deren Verwendung von Secrets und identifiziert diese.

\paragraph{Konkret möchten wir in Erfahrung bringen}
\begin{itemize}
    \item Um welche Secrets es sich handelt
    \item Welche Anforderungen diese Secrets verlangen (e.g. HMAC-SHA256)
    \item Ob eine Nutzerinteraktion notwendig ist, i.e. ob Anwender das Secret sehen und anpassen können müssen
\end{itemize}

\paragraph{Die folgenden Resultate werden erwartet:}
\begin{itemize}
    \item Welche Secrets gitb es?
    \item Pro Secret
        \begin{itemize}
              \item Benötigt dieses Secret ein spezifisches Format?
              \item Könnte ein solches Secret generiert werden?
              \item Muss dieses Secret angepasst werden können?
        \end{itemize}
    \item Abhängig von den Antworten zu den oberen Fragen soll zusätzlich beantwortet werden, ob es sinnvoll ist, verschiedene Arten von Secrets zu implementieren.
        Die Antwort ist zu begründen.
        Falls es Sinn macht, sind die Implementationsarten zu erstellen.
\end{itemize}

\subsection{Integration}\label{subsec:detailierte-aufgabenstellung-integration}
In einem nächsten Schritt soll für alle verschiedenen Secrets eine Integrationslösung erarbeitet und implementiert werden.
Secrets, welche keine Interaktion benötigen, sollen automatisch erstellt werden und für den Anwender unsichtbar sein.
Secrets, welche eine Interaktion benötigen, müssen abgefragt werden.
Darüber hinaus soll die Verwendung von Secrets im Getting Started Wizard vereinfacht werden, indem ein Anwender ein Secret im Modal aus einer Liste (Dropdown) auswählen kann.

\paragraph{Die folgenden funktionalen Anforderungen müssen erfüllt sein:}

\begin{enumerate}[label=A\arabic*:, leftmargin=*, itemsep=0.25em]
    \item Secrets werden automatisch zusammen mit dem Tenant erstellt und sind sofort für den Getting Started Wizard verfügbar
    \item Der Tenant darf nicht erstellt werden, wenn es einen Fehler beim Erstellen der Secrets gibt
    \item Secrets, welche nicht vom Benutzer überschrieben werden können, dürfen für den Benutzer auch nicht sichtbar sein
    \item Secrets müssen richtig in die Config Transformation integriert und für das Tenant IAM verwendet werden
    \item Die Inhalte der Secrets dürfen nicht für Anwender sichtbar sein, auch nicht in der Config (ZIP Download)
    \item Secret-Konfiguration im Getting Started Wizard
    \begin{enumerate}[label=A\arabic{enumi}\alph*:, leftmargin=*, itemsep=0.1em]
        \item Secrets, welche direkt im Getting Started Wizard konfiguriert werden müssen, sollen via Dropdown verfügbar sein
        \item Die Richtigkeit der Secret Referenz muss überprüft und sichergestellt werden
    \end{enumerate}
    \item Secrets müssen korrekt und vollständig in die Konfiguration eingefügt werden
\end{enumerate}

\section{Mittel und Methoden}\label{sec:mittel-und-methoden}

\paragraph{Technologien Backend}
\begin{itemize}
    \item Kotlin
    \item Spring
    \item Spring Events
    \item MongoDB
    \item Hashicorp Vault
    \item OpenAPI
\end{itemize}

\paragraph{Technologien Frontend}
\begin{itemize}
    \item React
    \item TypeScript
    \item React Hook From
\end{itemize}

\paragraph{Tools}
\begin{itemize}
    \item IntelliJ
    \item Git
    \item Gradle
    \item Windsurf
\end{itemize}

\paragraph{Architektur}
\begin{itemize}
    \item Microservices
    \item REST API
\end{itemize}


\section{Vorkenntnisse}\label{sec:vorkenntnisse}
\begin{itemize}
    \item Die Architektur, die Entwicklungsumgebung sowie unsere CI/CD Pipeline sind dem Lernenden bekannt.
    \item Er ist darüber hinaus auch mit den meisten Komponenten im Projekt bereits vertraut, da er bei vielen Features bereits mitwirken konnte.
    \item Er arbeitet seit ca. 12 Monaten in unserem Team.
\end{itemize}


\section{Vorarbeiten}\label{sec:vorarbeiten}
Es wurden keine spezifischen Vorarbeiten getätigt.
Da der Auszubildende bereits produktiv auf dem Projekt arbeitet, ist die Entwicklungsumgebung bereits vollständig aufgesetzt.

\section{Neue Lerninhalte}\label{sec:neue-lerninhalte}
\begin{itemize}
    \item Domain Design Secrets
    \item Secret Management
    \item Hashicorp Vault
\end{itemize}


\section{Arbeiten in den letzten 6 Monaten}\label{sec:arbeiten-in-den-letzten-6-monaten}
\begin{itemize}
    \item Features und Fixes im Airlock SaaS IAM
    \item Features und Fixes generell im Frontend sowie im Backend
    \item Features und Fixes im Getting Started Wizard
\end{itemize}
