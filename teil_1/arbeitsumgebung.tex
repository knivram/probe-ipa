\chapter{Arbeitsumgebung}\label{ch:arbeitsumgebung}
In diesem Kapitel wird die Arbeitsumgebung des Kandidaten während der Probe-IPA beschrieben.

\section{Arbeitsplatz}\label{sec:arbeitsplatz}
Der Kandidat arbeitet während seiner Probe-IPA an seinem üblichen Arbeitsplatz in einem Grossraumbüro.
Als Arbeitsgerät ist ein MacBook im Einsatz.
Dieses ist über eine Dockingstation mit zwei Monitoren und dem Netzwerk verbunden.
Tisch und Stuhl sind höhenverstellbar und ermöglichen es, die Position während der Arbeit regelmässig zu ändern.

In Abbildung \ref{fig:workspace} ist der Arbeitsplatz zu sehen.

\begin{figure}[H]
    \centering
    \includegraphics[width=0.8\textwidth]{ressourcen/placeholder}
    \caption{Arbeitsplatz des Kandidaten}\label{fig:workspace}
\end{figure}

\section{Verwendete Tools}\label{sec:verwendete-tools}
Die folgende Tabelle listet alle wichtigen Tools auf, mit denen der Kandidat während der Probe-IPA arbeitet.

\renewcommand{\arraystretch}{1.5}
\begin{longtable}{|p{.22\textwidth}|p{.40\textwidth}|p{.38\textwidth}|}
    \hline
    \textbf{Tool} & \textbf{Einsatzzweck} & \textbf{Link} \\ \hline
    IntelliJ IDEA & Entwicklungsumgebung für die Programmierung und das Schreiben der Dokumentation & \url{https://www.jetbrains.com/idea/} \\ \hline
    Windsurf-Plugin & KI-Unterstützung in der Entwicklungsumgebung & \url{https://windsurf.com/plugins} \\ \hline
    Git & Versionierung des Quellcodes und der IPA-Dokumentation & \url{https://git-scm.com/} \\ \hline
    GitLab & Speicherung des Quellcodes & \url{https://about.gitlab.com/} \\ \hline
    GitHub & Speicherung der Dokumentation & \url{https://github.com/} \\ \hline
    draw.io & Erstellung von Diagrammen & \url{https://www.drawio.com/} \\ \hline
    ChatGPT & KI-unterstützte Korrektur/Überarbeitung der Dokumentation & \url{https://chatgpt.com/overview} \\ \hline
    \caption{Verwendete Tools}\label{tab:used-tools}
\end{longtable}
\renewcommand{\arraystretch}{1}
\newpage
